\documentclass[a4paper,12pt]{amsart}

\input{preambolo}
\usepackage{kodi}

\usetikzlibrary{arrows}
\title{A categorical trinitarism}
\author{Moi}
\begin{document}
\maketitle
\epigraph{
\textgreek{
Ὡσαύτως καὶ αἱ τρεῖς ἡμέραι πρὸ τῶν φωστήρων γεγονυῖαι τύποι εἰσὶν τῆς τριάδος, τοῦ θεοῦ καὶ τοῦ λόγου αὐτοῦ καὶ τῆς σοφίας αὐτοῦ. τετάρτῳ δὲ τόπῳ ἐστὶν ἄνθρωπος ὁ προσδεὴς τοῦ φωτός, ἵνα ᾖ θεός, λόγος, σοφία, ἄνθρωπος. διὰ τοῦτο καὶ τετάρτῃ ἡμέρᾳ ἐγενήθησαν φωστῆρες.}}{Theophilus of Antioch, To Autolycus, \textsc{ii.xv}}
Every elementary categorical construction is subsumed by the notion of co/limit. This is not true any more for \emph{enriched} categories, where a more general notion is needed to capture the intrisic `two-dimensionality' of the theory.

The best we can do, in an enriched/higher setting, is to appeal the \textsc{kwc} \emph{trinitarism} that defines Kan extension, weighted co/limits and co/ends. The plan of this first introductory lecture is to give a minimal amount of definitions in order to establish a language which will be useful later on.

These concepts (co/ends, especially) offer powerful computational tools. The daunting task of learning abstract homotopy theory will be relieved letting these machineries formally produce fairly involved results.
\section{Kan extensions}
\begin{definition}
Let 
\[
\xymatrix{
	\C \ar[r]^F\ar[d]_G& \D \\
	\cate{E}
}
\]
be a diagram of categories and functors; a \emph{left Kan extension} of $F$ along $G$ is a functor, denoted $\Lan_GF$, endowed with a natural transformation $\eta\colon F \Rightarrow \Lan_GF\circ G$, initial among these natural transformations.

This means that whenever another natural transformation $\alpha \colon F \Rightarrow KG$ is given, there is a unique natural transformation $\bar\alpha \colon \Lan_GF\Rightarrow K$ such that $\alpha = \eta \circ \bar\alpha_G$.
\end{definition}
The situation is depicted in the following diagram:
\[
\xymatrix@R=1.5cm@C=1.5cm{
	\C \ar[r]^F \ar[d]_G & \D \\
	\cate{E} \ar[ur]|{\Lan_GF}\ar@/_2pc/[ur]_K
	\ar@{=>} (15,-15);(11.5,-11.5)
	\ar@{=>} (10,-2);(3,-10)
}
\]
We are mainly interested in a slightly less general definition, though. A \emph{pointwise} left Kan extension is a left Kan extension $\Lan_GF$ that is preserved by every representable functor. A rather common situation where a pointwise left Kan extension exists is when the codomain $\D$ of $F$ is cocomplete. In that case not only we find $\Lan_GF$ for \emph{every} $G$, but we are also able to characterize it with a more intuitive universal property: the Kan extensions can be constructed, over each object (a point, in some sense) of the domain category, using a suitable colimit.
\begin{proposition}
Let $G^*\colon [\C,\D]\to [\cate{E},\D]$ be the functor induced by precomposition with $G$: then $\Lan_GF$ is the functor that acts, on objects and morphisms, as
\[
e\mapsto \varinjlim_{(G\downarrow e)} Fc,
\]
where $(G\downarrow e)$ is the \emph{comma category} of arrows $Gc\to e$ and morphisms $c \to c'$ such that the obvious triangle commutes.

Moreover, the correspondence $\Lan_G(\firstblank)\colon [\C,\D]\leftarrow [\cate{E},\D]$ that sends $F\mapsto \Lan_GF$ is the left adjoint of $G^*$ that sends $H$ to $HG$.
\end{proposition}
\begin{proof}
A slightly more formal rephrasing of the above isomorphism is the following: consider the composition of functors
\[
\xymatrix{
	(G\downarrow e) \ar[r]^\Sigma & \C \ar[r]^F & \D
}
\]
then the colimit
\[
\varinjlim_{(c, \alpha : Gc \to e)\in(G\downarrow e)} F(\Sigma(c, \alpha))
\]
of this diagram on $\D$ is precisely the value of $\Lan_GF(e)$ which is then computed \emph{pointwise}.

To prove the statement,we build an explicit natural isomorphism
\[
\Nat(F,HG)\overset{\Psi}{\underset{\Phi}\leftrightarrows} \Nat(\Lan_GF,H).
\]
First of all we define a map $\Phi$, sending $\eta\colon F \to HG$ to the cocone with components
\[
\xymatrix{
	**[l] \hat{\eta}_{c,\alpha} \colon Fc \ar[r]^{\eta_c} & HGc \ar[r]^{H\alpha} & Hc
}
\]
This induces (via the universal property of the colimit) a unique map
\[
\varinjlim_{(G\downarrow e)} Fc \xto{\qquad \Phi(\eta)_e\qquad} He
\]
which is a natural transformation in $e$, since the square
\[
\xymatrix{
	\displaystyle \varinjlim_{(G\downarrow e)} Fc \ar[r]\ar[d]& He \ar[d]\\
	\displaystyle \varinjlim_{(G\downarrow e)} Fc \ar[r] & He'
}
\]
commutes for $f\colon e\to e'$.

On the other hand we define $\Psi$ taking the components $\beta_e \colon \beta\colon \varinjlim_{(G\downarrow e)} Fc \to He$ and considering the compositions
\[
\xymatrix@C=1.5cm{
	\varinjlim_{(G\downarrow x)} Fc \ar[r] & HGx \ar[dd]\\
	Fx\ar[dd]\ar[ur]\ar[u]^{\iota_{(x, 1_{Gx})}} & \\
	& HGy\\
	Fy \ar[ur] \ar[r]_{\iota_{(y, 1_{Gy})}}& \ar[u] \varinjlim_{(G\downarrow y)} Fc \\
}
\]
The inner square commutes, and proves naturality in $x\to y$; the reason is that it splits into two sub-diagrams, each of which commutes.
\end{proof}
\begin{exercise}
Dualize; define right Kan extensions $\Ran_GF$ and their universal property (how is the direction of 1-cells affected by the dualization?); define pointwise right Kan extensions and show a similar adjunction formula
\[
\Nat(HG,F)\cong \Nat(H, \Ran_GF).
\]
\end{exercise}
\subsection{Examples}
\begin{example}
Let $H\le G$ a group inclusion, regarded as a functor. The category of (left) $G$-sets is identified with the category of functors $[G,\Sets]$. 

There exist then functors
\[
npoiapofa
\]
given by Kan extension. Their action on objects and morphisms is given by
\[
fdsnpaia
\]
\end{example}
The following is a standard statement that is often subsumed by the sentence `any functor $F$ defined on presheaves on $\C$ is uniquely determined by its action on representables'.
\begin{example}
Let $F:\C \to \D$ be a functor from a \emph{small} to a \emph{cocomplete} category; then there is a unique way to define a functor $\bar F : [\C^\text{op}, \Sets] \to \D$ that commutes with colimits, and that has a right adjoint $N_F$ sending an object $d\in\D$ into the functor $c\mapsto \hom(Fc,d)$. 

This left adjoint sends a presheaf $P$ into the object $\Lan_yF(P)$ in $\D$.
\end{example}
\begin{proof}
It is enough to show that a natural transformation $\alpha\colon P \to N_F(d)$ determines a cocone for $(y\downarrow P) \to \C \overset{F}\to \D$: but from the commutativity of the square
\[
\xymatrix{
	c \ar[d]&  Pc & \hom(Fc,d)\\
	c' & Pc' & \hon(Fc',d)
}
\]
it follows that the triangle
\[
\xymatrix{
	Fc && Fc' \\
	&d&
}
\]
commutes, and then sending $(c, x\in Pc)$ into $\alpha_c(x) \colon Fc \to d$ determines a cocone for $FU$. This concludes the proof.
\end{proof}
\begin{exercise}
Show that $\Lan_1F\cong F$ and $\Ran_1F\cong F$, for every functor $F\colon \A\to \B$ (use the universal property).
\end{exercise}
\begin{exercise}
Show that if $F\colon \A\to\B$, $\A$ is small, $F$ is cocontinuous and $\B$ is cocomplete, then $\Lan_F1$ (where $1\colon \B\to\B$ is the identity functor) is a left adjoint for $F$.
\end{exercise}
\begin{exercise}
Use the universal property to show that $G\mapsto \Lan_G$ is a pseudofunctor, by showing that for $\A\xto{F}\B, \A\xto{G}\C\xto{H}\D$ there is a uniquely determined laxity cell for composition
\[
\Lan_H(\Lan_GF)\cong \Lan_{H\circ G}F.
\]
Dualize for right Kan extensions.
\end{exercise}
\section{Weighted co/limits}
The notion of weighted colimit generalizes the notion of colimit to an enriched setting, and provides a fairly formal approach to the issues posed by abstract homotopy theory; it is noticeable that the notion of \emph{homotopy colimit}\footnote{A homotopy colimit is a construction in a model category --not a real colimit-- that corrects the fact that the functor $\varinjlim$ does not preserve weak equivalences.} is captured, under mild assumptions on the model category, by a weighted colimit construction. Thus, the machinery produces a fairly explicit approach to the matter.

When looking for a classical (from now on, `conical') colimit of a diagram $F\colon J \to \C$, we basically are looking for a representative for the functor $a\mapsto \text{Cocones}(F,a)\cong \Nat(F,\Delta a)$, where $\Delta : \C \to \C^J$ is the `constant diagram' functor.

The latter object, that uniquely determines the object $\varinjlim F$, can be rewritten as
\[
\Nat(*, \hom(F,a))
\]
where now $\hom(F,a)$ is the functor $x\mapsto \hom(Fx,a)$ and $*$ is the terminal presheaf: then
\[
\hom(\varinjlim F,a)\cong \Nat(*, \hom(F,a)).
\]
A weighted colimit arises answering the following simple question: what happens if we replace the terminal presheaf $*$ with a more general $W\colon \cate{X}^\opp\to \Sets$, and look for a representative of the functor $a\mapsto \Nat(W, \hom(F,a)$? 

If we were able to solve this universal problem, we would obtain a natural isomorphism
\[
\hom(\varinjlim{}^W F,a)\cong \Nat(W, \hom(F,a)).
\]
for an object $\varinjlim{}^W F$ that we call the colimit of $F$ weighted by $W$.
\subsection{Examples}
\section{Co/ends}
\end{document}