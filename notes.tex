\documentclass[a4paper,12pt]{amsart}

% ########################################
% GENERAL PACKAGES
\usepackage[usenames,dvipsnames]{xcolor}


\usepackage{
   amssymb
  ,amsfonts
  ,url
  ,amsmath
  ,etex
  ,stmaryrd
  ,lettrine
  ,tikz
  ,esint
  ,epigraph
  ,manfnt
  ,xspace
  ,etoolbox
  ,microtype
  ,lmodern
  ,marginnote
  ,rotating
  ,datetime
  ,mathtools
  ,hyphenat
  ,newclude}

% \usetikzlibrary{kD}
\usetikzlibrary{matrix,calc,fit,shapes,positioning}

\usepackage[spanish, greek, english]{babel}
\usepackage[utf8]{inputenc}
\usepackage[T1]{fontenc}

\usepackage[all,2cell,cmtip]{xy}\UseAllTwocells

% ########################################

% ########################################
% STYLING
\setlength{\epigraphwidth}{0.65\textwidth}

\makeatletter
  \def\@cite#1#2{[\textbf{#1}\if@tempswa , #2\fi]}
  \def\@biblabel#1{[\textsf{#1}]}
\makeatother

\newlength{\seplen}
\setlength{\seplen}{5pt}
%
\newlength{\sepwid}
\setlength{\sepwid}{.4pt}
%
\def\firstblank{\,\rule{\seplen}{\sepwid}\,}
\def\secondblank{\firstblank\llap{\raisebox{2pt}{\firstblank}}}

\renewcommand{\textbf}[1]{\text{\fontseries{b}\selectfont{\upshape #1}}}
  \newcommand{\trans}[1]{\textcolor{red}{\bf #1}}
  \newcommand{\refbf}[1]{\textbf{\ref{#1}}}

\usepackage{enumitem}
\setlist[itemize]{noitemsep, topsep=2pt}
\setlist[enumerate]{label=(\oldstylenums{\arabic*}), noitemsep, topsep=2pt}


\usepackage[cal=boondoxo]{mathalfa}
% ########################################

% ########################################
% COMMANDS FOR TEXT
\def\mho{\rotatebox[origin=c]{180}{$\Omega$}}

\def\proP{\textsc{p}}
\def\proQ{\textsc{q}}
\def\proH{\textsc{h}}
\def\proL{\textsc{l}}
\def\proK{\textsc{k}}
\def\proG{\textsc{g}}

\providecommand{\abbrv}[1]{#1.\@\xspace}
  \providecommand{\ie}{\abbrv{i.e}}
  \providecommand{\etc}{\abbrv{etc}}
  \providecommand{\prof}{\abbrv{prof}}
  \providecommand{\viz}{\abbrv{viz}}
  \providecommand{\eg}{\abbrv{e.g}}
  \providecommand{\achap}{\abbrv{Ch}}
  \providecommand{\adef}{\abbrv{Def}}
  \providecommand{\aprop}{\abbrv{Prop}}
  \providecommand{\athm}{\abbrv{Thm}}

\usepackage{CJKutf8}

% The yoneda embedding: the letter "yo" in hiragana
\newcommand{\yon}{\text{\begin{CJK}{UTF8}{min}よ\end{CJK}}}

\newcommand{\wk}{\textsc{wk}}
\newcommand{\cof}{\textsc{cof}}
\newcommand{\fib}{\textsc{fib}}

\def\itsnonsense{\upeyes}

\setcounter{tocdepth}{2} 
\usepackage[titletoc]{appendix}
\usepackage[
       a4paper
      ,bottom=6cm
      ,left=4.2cm
      ,right=4.2cm
      ,top=4.5cm]{geometry}
% ########################################

% ########################################
% COMMANDS FOR MATH

\usepackage{amsthm}

\newtheoremstyle{reference}
   {}                
   {}                
   {}              
   {}                      
   {\fontseries{b}\selectfont}              
   {:}                     
   {.2em}                  
   {\thmname{#1}           
    \thmnumber{#2}         
    \thmnote{{\sc [#3]}}} 

\theoremstyle{reference}
  \newtheorem{theorem}{Theorem}[section]
  \newtheorem{lemma}[theorem]{Lemma}
  \newtheorem{proposition}[theorem]{Proposition}
  \newtheorem{example}[theorem]{Example}
  \newtheorem{exercise}[theorem]{Exercise}
  \newtheorem{remark}[theorem]{Remark}
  \newtheorem{definition}[theorem]{Definition}
  \newtheorem{corollary}[theorem]{Corollary}
  \newtheorem{notat}[theorem]{Notation}
  \newtheorem*{acknowledgements}{Acknowledgements}
  \newtheorem{scholium}[theorem]{Scholium}
  \newtheorem{counterex}[theorem]{Counterexample}
  % starred
  \newtheorem*{theorem*}{Theorem}
  \newtheorem*{lemma*}{Lemma}
  \newtheorem*{proposition*}{Proposition}
  \newtheorem*{example*}{Example}
  \newtheorem*{exercise*}{Exercise}
  \newtheorem*{remark*}{Remark}
  \newtheorem*{definition*}{Definition}
  \newtheorem*{corollary*}{Corollary}
  \newtheorem*{notat*}{Notation}
  \newtheorem*{scholium*}{Scholium}
  \newtheorem*{counterex*}{Counterexample}

\newcommand{\bDelta}{\pmb\Delta}
\newcommand{\Dist}{\cate{Relt}}
\newcommand{\varsSet}{\widehat{\bDelta}}
\newcommand{\simplex}[1]{{\Delta[#1]}}
\newcommand{\xto}[2][]{\xrightarrow[#1]{#2}}
\newcommand{\xot}[2][]{\xleftarrow[#1]{#2}}
\newcommand{\bsmat}{\left[\begin{smallmatrix}}
\newcommand{\esmat}{\end{smallmatrix}\right]}
\renewcommand{\phi}{\varphi}
\newcommand{\var}[3][]{
  \left[\begin{smallmatrix} #2 \\ 
  #1\downarrow \\ #3 
  \end{smallmatrix}\right]}
\newcommand{\laxto}{\dashrightarrow}
\newcommand{\ccirc}{\diamond}

\DeclareMathOperator{\id}{id}
\DeclareMathOperator{\Lan}{Lan}
\DeclareMathOperator{\Ran}{Ran}
\DeclareMathOperator{\hoLan}{hoLan}
\DeclareMathOperator{\hoRan}{hoRan}
\DeclareMathOperator{\Rift}{Rift}
\DeclareMathOperator{\Lift}{Lift}
\DeclareMathOperator{\ob}{Ob}
\newcommand{\opp}{\text{op}}
\newcommand{\oo}{\text{oo}}
\newcommand{\Ex}{\text{Ex}}

\newcommand{\dashlim}{\text{-lim }}
\providecommand{\defequal}  {\triangleq}

\newcommand{\cate}[1]{\text{\fontseries{b}\selectfont{#1}}}
  \newcommand{\A}     {\cate{A}}
  \newcommand{\B}     {\cate{B}}
  \newcommand{\C}     {\cate{C}}
  \newcommand{\Cat}   {\cate{Cat}}
  \newcommand{\Gpd}   {\cate{Gpd}}
  \newcommand{\D}     {\cate{D}}
  \newcommand{\K}     {\cate{K}}
  \renewcommand{\P}   {\cate{P}}
  \newcommand{\catS}  {\cate{S}}
  \newcommand{\Sets}  {\cate{Sets}}
  \newcommand{\Set}   {\cate{Sets}}
  \newcommand{\sSet}  {\cate{sSet}}

  \newcommand{\N}{\mathbb{N}}
  \newcommand{\Z}{\mathbb{Z}}

  \newcommand{\Top}   {\cate{Top}}
  \newcommand{\E}     {\mathcal{E}}
  \renewcommand{\L}   {\mathcal{L}}
  \newcommand{\M}     {\mathcal{M}}
  \newcommand{\R}     {\mathcal{R}}
  \newcommand{\V}     {\mathcal{V}}
  \newcommand{\tee}     {\mathfrak{t}}
  \newcommand{\T}     {\mathfrak{T}}
  \newcommand{\Y}     {\mathfrak{Y}}
  \newcommand{\W}     {\mathfrak{W}}
  \newcommand{\VCat}  {\V\text{-}\Cat}

\def\Nat{\textsf{Nat}}
\def\Fun{\textsf{Fun}}
\def\LNat{\textsf{LNat}}
\def\bemo{\flat}
\def\diesis{\sharp}
\newcommand{\pto}{\leadsto}
\newcommand{\wlim}[1]{{\textstyle \varprojlim^{#1}}}
\newcommand{\wcolim}[1]{{\textstyle \varinjlim^{#1}}}
\newcommand{\hocolim}{\underrightarrow{\text{holim}}\,}
\newcommand{\holim}{\underleftarrow{\text{holim}}\,}
\newcommand{\elts}[2]{#1\!\int\! #2}
\newcommand{\tw}{\textsc{tw}}

\newcommand{\twoint}{\sqint}
\newcommand{\infint}{\oint}

% ########################################

% ########################################
% ALTRO
% DECOMMENTA X HYPHENS
% \tracingparagraphs=1
% \tracingonline=1
% \input{ipenate}

\newcommand{\arXivPreprint}[1]{arXiv preprint \href{http://arxiv.org/abs/#1}{arXiv:#1}}


\newcommand{\standard}{
\draw[ultra thin] (0.25,0) -- (1.75,0) -- (1.75,1) -- (0.25,1) -- cycle;
\draw (.5,0) -- (.5,1);
\draw (1,0) .. controls (1,.5) and (1.5,.5) .. (1.5,0);
\draw (1,1) .. controls  (1,.5) and (1.5,.5) .. (1.5,1);
}

\newcommand{\standardbis}[3]{
\draw[ultra thin] (0.25,0) -- (1.75,0) -- (1.75,1) -- (0.25,1) -- cycle;
\draw[#1] (.5,0) -- (.5,1);
\draw[#2] (1,0) .. controls (1,.5) and (1.5,.5) .. (1.5,0);
\draw[#3] (1,1) .. controls  (1,.5) and (1.5,.5) .. (1.5,1);
}

\newcommand{\first}[1]{\draw[ultra thin] (0.25,0) -- (1.75,0) -- (1.75,1) -- (0.25,1) -- cycle;
\draw (.5,0) -- (.5,1);
\draw[xshift=.5cm] (.5,0)  -- (.5,1);
\draw[xshift=1cm] (.5,0)-- (.5,1);
\filldraw[lightgray!70] (.5,.5) circle (6pt) node[black] {$ #1 $};}
\newcommand{\second}[1]{\draw[ultra thin] (0.25,0) -- (1.75,0) -- (1.75,1) -- (0.25,1) -- cycle;
\draw (.5,0) -- (.5,1);
\draw[xshift=.5cm] (.5,0)  -- (.5,1);
\draw[xshift=1cm] (.5,0)-- (.5,1);
\filldraw[lightgray!70] (1,.5) circle (6pt) node[black] {$ #1 $};}
\newcommand{\third}[1]{\draw[ultra thin] (0.25,0) -- (1.75,0) -- (1.75,1) -- (0.25,1) -- cycle;
\draw (.5,0) -- (.5,1);
\draw[xshift=.5cm] (.5,0)  -- (.5,1);
\draw[xshift=1cm] (.5,0)-- (.5,1);
\filldraw[lightgray!70] (1.5,.5) circle (6pt) node[black] {$ #1 $};}


%%%%%%%%%%%%%%%%%%

\newcommand{\upeyes}{%
  \begin{tikzpicture}
  \draw circle (3pt);
  \draw[fill,yshift=1.5pt] circle (1.5pt);
  \begin{scope}[xshift=7pt]
  \draw circle (3pt);
  \draw[fill,yshift=1.5pt] circle (1.5pt);
  \end{scope}
  \end{tikzpicture}%
}

\newcommand{\downeyes}{%
  \begin{tikzpicture}
  \draw circle (3pt);
  \draw[fill,yshift=-1.5pt] circle (1.5pt);
  \begin{scope}[xshift=7pt]
  \draw circle (3pt);
  \draw[fill,yshift=-1.5pt] circle (1.5pt);
  \end{scope}
  \end{tikzpicture}%
}

\newcommand{\righteyes}{%
  \begin{tikzpicture}
  \draw circle (3pt);
  \draw[fill,xshift=1.5pt] circle (1.5pt);
  \begin{scope}[xshift=7pt]
  \draw circle (3pt);
  \draw[fill,xshift=1.5pt] circle (1.5pt);
  \end{scope}
  \end{tikzpicture}%
}

\newcommand{\awful}{%
   \begin{tikzpicture}
   \draw circle (3pt);
   \draw[fill] circle (1.5pt);
   \begin{scope}[xshift=7pt]
   \draw circle (3pt);
   \draw[fill] circle (1.5pt);
   \end{scope}
   \end{tikzpicture}%
}


% ########################################

% ########################################
% ALCUNE COSE VANNO SEMPRE IN FONDO
\usepackage{hyperref}
\hypersetup{%
  pdftoolbar=   true,
  pdfmenubar=   true,
  pdffitwindow= true,
  pdftitle=     {notes},
  pdfauthor=    {F. Loregian},
  colorlinks=   true,
  linkcolor=    black,
  citecolor=    blue!50}

% ########################################

\def\[{\begin{equation}}
\def\]{\end{equation}}

\usepackage[framemethod=TikZ]{mdframed}

\makeatletter
\def\exerciseboxtag{
  \useasboundingbox (P) rectangle (P);
  \node at (P-|O) [
      draw=black!70,
      line width=.4pt,
      fill=black!2,
      rectangle,
      rounded corners=2pt,
      inner sep=.3em,
      anchor=west,
      xshift=3em,
      font=\mdf@frametitlefont,
      execute at begin node=\strut~,
      execute at end node=~
  ] {Exercises for \S\thesection};
}
\makeatother

\mdfdefinestyle{exstyle}{%
  % these two account for spacing around the tag:
  skipabove=1.6em, innertopmargin=1.6em,
  middlelinewidth=.4pt,
  roundcorner=1.6pt,
  linecolor=black!80,
  backgroundcolor=black!2,
  frametitlefont=\bfseries,
  % settings={\global\stepcounter{question}},
  singleextra={\exerciseboxtag},
  firstextra={\exerciseboxtag}
}

\newenvironment{exerciseset}
  {\begin{mdframed}[style=exstyle]\footnotesize}
  {\end{mdframed}}

\allowdisplaybreaks

\usepackage{enumitem}
\newlist{exercisepoints}{enumerate}{1}
\setlist[exercisepoints]{
  label={\textbf{E\arabic*}},
  leftmargin=2em,
  % before=\raggedright
}
\usepackage{kodi}
\def\yon{y}
\usetikzlibrary{arrows}
\title{A categorical trinitarism}
\author{Moi}
\begin{document}
\maketitle
\epigraph{
\textgreek{
Ὡσαύτως καὶ αἱ τρεῖς ἡμέραι πρὸ τῶν φωστήρων γεγονυῖαι τύποι εἰσὶν τῆς τριάδος, τοῦ θεοῦ καὶ τοῦ λόγου αὐτοῦ καὶ τῆς σοφίας αὐτοῦ. τετάρτῳ δὲ τόπῳ ἐστὶν ἄνθρωπος ὁ προσδεὴς τοῦ φωτός, ἵνα ᾖ θεός, λόγος, σοφία, ἄνθρωπος. διὰ τοῦτο καὶ τετάρτῃ ἡμέρᾳ ἐγενήθησαν φωστῆρες.}}{Theophilus of Antioch, To Autolycus, \textsc{ii.xv}}
Every elementary categorical construction is subsumed by the notion of co/limit. This is not true any more for \emph{enriched} categories, where a more general notion is needed to capture the intrisic `two-dimensionality' of the theory.

The best we can do, in an enriched/higher setting, is to appeal the \textsc{kwc} \emph{trinitarism} that defines Kan extension, weighted co/limits and co/ends. The plan of this first introductory lecture is to give a minimal amount of definitions in order to establish a language which will be useful later on.

These concepts (co/ends, especially) offer powerful computational tools. The daunting task of learning abstract homotopy theory will be relieved letting these machineries formally produce fairly involved results.
\section{Kan extensions}
\begin{definition}
Let 
\[
\xymatrix{
	\C \ar[r]^F\ar[d]_G& \D \\
	\cate{E}
}
\]
be a diagram of categories and functors; a \emph{left Kan extension} of $F$ along $G$ is a functor, denoted $\Lan_GF$, endowed with a natural transformation $\eta\colon F \Rightarrow \Lan_GF\circ G$, initial among these natural transformations.

This means that whenever another natural transformation $\alpha \colon F \Rightarrow KG$ is given, there is a unique natural transformation $\bar\alpha \colon \Lan_GF\Rightarrow K$ such that $\alpha = \eta \circ \bar\alpha_G$.
\end{definition}
The situation is depicted in the following diagram:
\[
\xymatrix@R=1.5cm@C=1.5cm{
	\C \ar[r]^F \ar[d]_G & \D \\
	\cate{E} \ar[ur]|{\Lan_GF}\ar@/_2pc/[ur]_K
	\ar@{=>} (15,-15);(11.5,-11.5)
	\ar@{=>} (10,-2);(3,-10)
}
\]
We are mainly interested in a slightly less general definition, though. A \emph{pointwise} left Kan extension is a left Kan extension $\Lan_GF$ that is preserved by every representable functor. A rather common situation where a pointwise left Kan extension exists is when the codomain $\D$ of $F$ is cocomplete. In that case not only we find $\Lan_GF$ for \emph{every} $G$, but we are also able to characterize it with a more intuitive universal property: the Kan extensions can be constructed, over each object (a point, in some sense) of the domain category, using a suitable colimit.
\begin{proposition}
Let $G^*\colon [\C,\D]\to [\cate{E},\D]$ be the functor induced by precomposition with $G$: then $\Lan_GF$ is the functor that acts, on objects and morphisms, as
\[
e\mapsto \varinjlim_{(G\downarrow e)} Fc,
\]
where $(G\downarrow e)$ is the \emph{comma category} of arrows $Gc\to e$ and morphisms $c \to c'$ such that the obvious triangle commutes.

Moreover, the correspondence $\Lan_G(\firstblank)\colon [\C,\D]\leftarrow [\cate{E},\D]$ that sends $F\mapsto \Lan_GF$ is the left adjoint of $G^*$ that sends $H$ to $HG$.
\end{proposition}
\begin{proof}
A slightly more formal rephrasing of the above isomorphism is the following: consider the composition of functors
\[
\xymatrix{
	(G\downarrow e) \ar[r]^\Sigma & \C \ar[r]^F & \D
}
\]
then the colimit
\[
\varinjlim_{(c, \alpha : Gc \to e)\in(G\downarrow e)} F(\Sigma(c, \alpha))
\]
of this diagram on $\D$ is precisely the value of $\Lan_GF(e)$ which is then computed \emph{pointwise}.

To prove the statement,we build an explicit natural isomorphism
\[
\Nat(F,HG)\overset{\Psi}{\underset{\Phi}\leftrightarrows} \Nat(\Lan_GF,H).
\]
First of all we define a map $\Phi$, sending $\eta\colon F \to HG$ to the cocone with components
\[
\xymatrix{
	**[l] \hat{\eta}_{c,\alpha} \colon Fc \ar[r]^{\eta_c} & HGc \ar[r]^{H\alpha} & Hc
}
\]
This induces (via the universal property of the colimit) a unique map
\[
\varinjlim_{(G\downarrow e)} Fc \xto{\qquad \Phi(\eta)_e\qquad} He
\]
which is a natural transformation in $e$, since the square
\[
\xymatrix{
	\displaystyle \varinjlim_{(G\downarrow e)} Fc \ar[r]\ar[d]& He \ar[d]\\
	\displaystyle \varinjlim_{(G\downarrow e)} Fc \ar[r] & He'
}
\]
commutes for $f\colon e\to e'$.

On the other hand we define $\Psi$ taking the components $\beta_e \colon \beta\colon \varinjlim_{(G\downarrow e)} Fc \to He$ and considering the compositions
\[
\xymatrix@C=2cm{
	Fx\ar[r]^{\iota_{(x, 1_{Gx})}}\ar[d] & \displaystyle\varinjlim_{(G\downarrow x)} Fc \ar[r]\ar[d] & HGx \ar[d] \\
	Fy \ar[r]_{\iota_{(y, 1_{Gy})}}& \displaystyle\varinjlim_{(G\downarrow y)} Fc \ar[r] & HGy 
}
% \xymatrix@C=1.5cm{
% 	\varinjlim_{(G\downarrow x)} Fc \ar[r] & HGx \ar[dd]\\
% 	Fx\ar[dd]\ar[ur]\ar[u]^{\iota_{(x, 1_{Gx})}} & \\
% 	& HGy\\
% 	Fy \ar[ur] \ar[r]_{\iota_{(y, 1_{Gy})}}& \ar[u] \varinjlim_{(G\downarrow y)} Fc \\
% }
\]
The inner square commutes, and proves naturality in $x\to y$; the reason is that it splits into two sub-diagrams, each of which commutes.
\end{proof}
\begin{exercise}
Dualize; define right Kan extensions $\Ran_GF$ and their universal property (how is the direction of 1-cells affected by the dualization?); define pointwise right Kan extensions and show a similar adjunction formula
\[
\Nat(HG,F)\cong \Nat(H, \Ran_GF).
\]
\end{exercise}
\subsection{Examples}
\begin{example}
Let $\iota\colon H\le G$ a group inclusion, regarded as a functor. The category of (left) $G$-sets is identified with the category of functors $[G,\cate{Vect}_k]$. 

There exist then functors
\[
\xymatrix{
	[G,\cate{Vect}_k] \ar[r]|{\iota^*} & [H,\cate{Vect}_k] \ar@<6pt>[l]^{\Lan_\iota}\ar@<-6pt>[l]_{\Ran_\iota}
}
\]
given by Kan extension. Their action on objects and morphisms is given by
\begin{gather*}
V \mapsto k[G]\otimes_{k[H]}V\\
W\mapsto \hom_{k[H]}(k[G],W)
\end{gather*}
(the group algebra $k[G]$ is the free $k$-algebra on group elements of $G$, with multiplication $\sum_g a_g \cdot \sum_{g'} b_{g'} = \sum_h\sum_{gg'=h}a_g b_{g'}$, and similarly for $H$).
\end{example}
\begin{exercise}
Prove the above statement, with the help of the following characterization for the two functors above:
\begin{itemize}
	\item There is an equalizer
	\[
	\xymatrix{
		\hom_{k[H]}(k[G],W) \ar[r]& \hom_k(k[G],V) \ar@<6pt>[r]^(.4){\hom(1,h)}\ar@<-6pt>[r]_(.4){\hom(h,1)} & \hom_k(k[G],V)
	}
	\]
	\item There is a coequalizer
	\[
	\xymatrix{
		**[l]\bigoplus_{h\in H}k[G]\otimes_k V \ar@<6pt>[r]^{1\otimes h}\ar@<-6pt>[r]_{h\otimes 1} & k[G]\otimes_k V \ar[r] & k[G]\otimes_{k[H]}V.
	}
	\]
\end{itemize}
\end{exercise}
The following is a standard statement that is often subsumed by the sentence `any functor $F$ defined on presheaves on $\C$ is uniquely determined by its action on representables'.
\begin{example}
Let $F:\C \to \D$ be a functor from a \emph{small} to a \emph{cocomplete} category; then there is a unique way to define a functor $\bar F : [\C^\text{op}, \Sets] \to \D$ that commutes with colimits, and that has a right adjoint $N_F$ sending an object $d\in\D$ into the functor $c\mapsto \hom(Fc,d)$. 

This left adjoint sends a presheaf $P$ into the object $\Lan_yF(P)$ in $\D$.
\end{example}
\begin{proof}
It is enough to show that a natural transformation $\alpha\colon P \to N_F(d)$ determines a cocone for $(y\downarrow P) \to \C \overset{F}\to \D$: but from the commutativity of the square
\[
\xymatrix{
	c \ar[d]_\varphi&  Pc \ar[r]& \hom(Fc,d)\\
	c' & Pc' \ar[u]^{P\varphi}\ar[r]& \hom(Fc',d)\ar[u]_{\hom(F\varphi,d)}
}
\]
it follows that the triangle
\[
\xymatrix{
	Fc \ar[rr]^{F\varphi}\ar[dr]_\alpha&& Fc'\ar[dl]^{\alpha'} \\
	&d&
}
\]
commutes, and then sending $(c, x\in Pc)$ into $\alpha_c(x) \colon Fc \to d$ determines a cocone for $FU$. This concludes the proof, once we show that this is a bijection; of course it is.
\end{proof}
\begin{exercise}
Show that the above construction determines an equivalence of categories 
\[
\Cat(\C,\D)\xto{\qquad\text{Yan}\qquad} \cate{Adj}_\textsc{l}([\C^\opp,\Sets],\D)
\]
if morphisms on the right are natural transformations between lef adjoints (hint: simply show that the adjunction $\Lan_y\dashv y^*$ is an equivalence of categories).
\end{exercise}
\begin{exercise}
Shape a dual statement regarding a universal property of the contravariant Yoneda embedding $\C^\opp\xto{\;y\;}[\C, \Sets]^\opp$. Is it really an independent theorem?
\end{exercise}
\begin{exercise}
Show that $\Lan_1F\cong F$ and $\Ran_1F\cong F$, for every functor $F\colon \A\to \B$ (use the universal property).
\end{exercise}
\begin{exercise}
Show that if $F\colon \A\to\B$ is a functor between small categories, then the following conditions are equivalent:
\begin{itemize}
	\item $F$ has a right adjoint;
	\item $\Lan_F1$ exists and there is an isomorphism $\Lan_FF\cong F$;
	\item $\Lan_F1$ exists and there is an isomorphism $\Lan_FL\cong L\circ \Lan_F1$ for any functor $L\colon \A\to \C$.
\end{itemize}
\end{exercise}
\begin{exercise}
Use the universal property to show that $G\mapsto \Lan_G$ is a pseudofunctor, by showing that for $\A\xto{F}\B, \A\xto{G}\C\xto{H}\D$ there is a uniquely determined laxity cell for composition
\[
\Lan_H(\Lan_GF)\cong \Lan_{H\circ G}F.
\]
Dualize for right Kan extensions.
\end{exercise}
\section{Weighted co/limits}
The notion of weighted colimit generalizes the notion of colimit to an enriched setting, and provides a fairly formal approach to the issues posed by abstract homotopy theory; it is noticeable that the notion of \emph{homotopy colimit}\footnote{A homotopy colimit is a construction in a model category --not a real colimit-- that corrects the fact that the functor $\varinjlim$ does not preserve weak equivalences.} is captured, under mild assumptions on the model category, by a weighted colimit construction. Thus, the machinery produces a fairly explicit approach to the matter.

When looking for a classical (from now on, `conical') colimit of a diagram $F\colon J \to \C$, we basically are looking for a representative for the functor $a\mapsto \text{Cocones}(F,a)\cong \Nat(F,\Delta a)$, where $\Delta : \C \to \C^J$ is the `constant diagram' functor.

The latter object, that uniquely determines the object $\varinjlim F$, can be rewritten as
\[
\Nat(*, \hom(F,a))
\]
where now $\hom(F,a)$ is the functor $x\mapsto \hom(Fx,a)$ and $*$ is the terminal presheaf: then
\[
\hom(\varinjlim F,a)\cong \Nat(*, \hom(F,a)).
\]
A weighted colimit arises answering the following simple question: what happens if we replace the terminal presheaf $*$ with a more general $W\colon \cate{X}^\opp\to \Sets$, and look for a representative of the functor $a\mapsto \Nat(W, \hom(F,a)$? 

If we were able to solve this universal problem, we would obtain a natural isomorphism
\[
\hom(\varinjlim{}^W F,a)\cong \Nat(W, \hom(F,a)).
\]
for an object $\varinjlim{}^W F$ that we call the colimit of $F$ weighted by $W$. A useful shorthand to denote the colimit of $F$ weighted by $W$, alternative to the ugly $\varinjlim{}^W F$, is $W\cdot F$, in such a way that its universal property is written
\[
\hom(W\cdot F,a)\cong \Nat(W, \hom(F,a)).
\]
\subsection{Examples}
\begin{example}
Let $W \colon * \to \Sets$ be the choice of a set $X$; let $F \colon * \to \C$ be the choice of an object $c$. A natural transformation $W\Rightarrow \hom(F,a)$ consists of a function of sets $X \to \hom(c,a)$, so that from the isomorphism
\[
\hom(W\cdot F,a)\cong \Sets(X, \hom(c,a))
\]
and the uniqueness of adjoints we deduce that 
\[\textstyle
W \cdot F = X \cdot c \cong \coprod_{x\in X} c
\]
\end{example}
\begin{exercise}
Generalize to the case where $W \colon J \to \Sets$ is a parametric family of sets, and $F\colon J \to \C$ a parametric family of objects.
\end{exercise}
\begin{example}
Let $W\colon \{0\to 1\}\to \Sets$ be the choice of a function of sets $f : X\leftarrow Y$, and $F\colon \{0\to 1\}\to \C$ be the choice of a morphism $u : c\to c'$ in $\C$. Then a natural transformation $W\Rightarrow \hom(F,a)$ consists of a pair of functions $\alpha_\epsilon$ ($\epsilon=0,1$) such that
\[
\xymatrix@C=2cm{
	X \ar[r]^{\alpha_0}& \hom(c,a)\\
	Y \ar[u]\ar[r]_{\alpha_1}& \hom(c',a)\ar[u]
}
\]
Show that $W\cdot F$ fits into a pushout diagram
\[
\xymatrix{
	Y\cdot c \ar@{}[dr]|\ulcorner \ar[r]\ar[d] & Y\cdot c'\ar[d]\\
	X \cdot c \ar[r] & W\cdot F.
}
\]
\end{example}
\begin{definition}\label{eltsf}
Let $W\colon \C\to \Sets$ be a functor; the \emph{category of elements} $\elts{\C}{W}$ of $W$ is the category having objects the pairs $(c\in\C, u\in Wc)$, and morphisms $(c,u)\to (c',v)$ those $f\in \C(c,c')$ such that $W(f)(u)=v$.
\end{definition}
\begin{notat}
The exotic notation ``$\elts{\C}{W}$'' for the category of elements of $W$ comes from the wondrous paper \cite{Graya}.
\end{notat}
\begin{proposition}
The category $\elts{\C}{W}$ can be equivalently characterized as
\begin{itemize}
\item The category which results from the pullback 
\[
\xymatrix{
  \elts{\C}{W}\ar[r]\ar[d]\ar@{}[dr]|\lrcorner & \Sets_* \ar[d]^U \\
  \C \ar[r]_W & \Sets
}
\]
where $U\colon \Sets_*\to\Sets$ is the forgetful functor which sends a pointed set to its underlying set;
\item The comma category of the cospan $\{*\}\to \Sets \xot{W} \C$, where $\{*\}\to \Sets$ chooses the terminal object of $\Sets$;
\item The opposite of the comma category $(\yon\downarrow \lceil W\rceil)$, where $\lceil W\rceil\colon \{*\}\to [\C, \Sets]$ is the \emph{name} of the functor $W$, \ie the unique functor choosing the presheaf $W\in [\C, \Sets]$.
% \begin{center}
% \includegraphics[scale=1]{figures/fig5}
% \end{center}
\end{itemize}
\end{proposition}
\begin{proposition}\label{fibelem}
The category of elements $\elts{\C}{W}$ of a functor $W\colon \C\to \Sets$ comes equipped with a canonical ``Grothendieck fibration'' to the domain of $W$, which we denote $\Sigma\colon \elts{\C}{W}\to \C$, defined forgetting the distinguished element $u\in Wc$. 
\end{proposition}
\begin{remark}[\righteyes, The Grothendieck construction trivializes weights]
The definition of weighted co/limit can be extended in the case $F\colon \C\to\A$ is a $\V$-enriched functor between $\V$-categories, and $W\colon \C\to\V$ is a $\V$-co/presheaf; this is the setting where the notion acquires a supremacy over the ``classical'' one (where the weight is the terminal presheaf). 

When $\V=\Sets$, indeed, the construction sending a (co)presheaf into its category of elements turns out to trivialize the theory of weighted limits: as the following discussion shows, in such a situation every weighted limit can be expressed as a `classical' (we prefer to call them \emph{conical}, due to the shape of the weight) limit.
\end{remark}
\begin{proposition}[$\Sets$-weighted limits are limits]\label{elementi}
As shown in Prop\@. \refbf{fibelem}, the category $\elts{\C}{W}$ comes equipped with a fibration $\Sigma\colon \elts{\C}{W}\to\C$; this is such that for any functor $F\colon \C\to \A$ one has 
\[\wlim{W}F\cong \varprojlim_{(c,x)\in \elts{\C}{W}} F\circ \Sigma.\]
\end{proposition}
\begin{proof}
We show that the two objects share the same universal property.
\end{proof}
\section{Co/ends}
\end{document}